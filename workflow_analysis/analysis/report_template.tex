\documentclass{article}

\begin{document}
\section{Simulation Setup}
\subsection{Starting Configuration}
% In this section we discuss how the starting configuration was made
%{{starting_config_comment%}}

\subsection{Experimental Parameters}
  The PDF of the %may want an if statment here
  final structure was generated using Eqn. (\ref{Eq:Gdef}) with a step of
    $\delta R=%{{experiment['rstep']%}} $~\AA,
    $R_\mathrm{min}=%{{experiment['rmin']%}} $~\AA$,
    $R_\mathrm{min}=%{{experiment['rmax']%}} $~\AA$,
    Q resolution, $\delta Q= %{{experiment['qbin']%}}$,
    $Q_\mathrm{min}=%{{experiment['qmin']%}} $~\AA$^{-1}$,
    $Q_\mathrm{max}=%{{experiment['qmax']%}}$~ \AA$^{-1}$

\subsection{Simulation Parameters}
The HMC simulation was run with
$N=%{{ensemble['iterations']%}}$
iterations, a target acceptance rate of
%{{ensemble['target_acceptance']%}},
and an average starting momentum for each NUTS iteration  of
%{{ensemble['temperature'] * 10%}} eVfs/\AA.

\subsection{PES}
Each simulation used both the repulsive and attractive spring potentials,
with $\kappa=%{{spring_kwargs['k']%}}$ eV/\AA
and a threshold matching the PDF $R_\mathrm{max}$ and $R_\mathrm{min}$ respectively.
$\lambda=$ %{{pdf_kwargs['conv']%}} eV was used as conversion factor for
%{{potential%}}.
\section{Simulation Results}
\begin{figure}
    \def \localimgpath }}
    \centering
            \begin{subfigure}{.33\textwidth}
                \includegraphics[width=2 in]{"\localimgpath_target"}
        \end{subfigure}
        \begin{subfigure}{.33\textwidth}
                \includegraphics[width=2 in]{"\localimgpath_min"}
        \end{subfigure}
        \\
        \begin{subfigure}{.33\textwidth}
                \includegraphics[width=2 in]{"\localimgpath_pdf"}
        \end{subfigure}
        \begin{subfigure}{.33\textwidth}
                \includegraphics[width=2 in]{"\localimgpath_coord"}
        \end{subfigure}
        \\
        \begin{subfigure}[b]{.33\textwidth}
                \includegraphics[width=2 in]{"\localimgpath_angle"}
        \end{subfigure}
        \begin{subfigure}[b]{.33\textwidth}
                \includegraphics[width=2 in]{"\localimgpath_rbonds"}
        \end{subfigure}
  \caption{From the top left: the target structure,
  the final structural solution,
  the comparison of PDFs,
  the coordination number distribution,
  the radial bond distribution, and
  the bond angle distribution.}
        \label{fig: %{{simulation_name%}}}
\end{figure}
The starting %{{potential%}} was %{{results['start_potential']%}} with a final
%{{potential%}} of %{{results['finish_potential']%}}
\end{document}